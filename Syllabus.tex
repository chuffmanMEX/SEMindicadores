\documentclass[11pt,]{article}
\usepackage{lmodern}
\usepackage{amssymb,amsmath}
\usepackage{ifxetex,ifluatex}
\usepackage{fixltx2e} % provides \textsubscript
\ifnum 0\ifxetex 1\fi\ifluatex 1\fi=0 % if pdftex
  \usepackage[T1]{fontenc}
  \usepackage[utf8]{inputenc}
\else % if luatex or xelatex
  \ifxetex
    \usepackage{mathspec}
  \else
    \usepackage{fontspec}
  \fi
  \defaultfontfeatures{Ligatures=TeX,Scale=MatchLowercase}
\fi
% use upquote if available, for straight quotes in verbatim environments
\IfFileExists{upquote.sty}{\usepackage{upquote}}{}
% use microtype if available
\IfFileExists{microtype.sty}{%
\usepackage{microtype}
\UseMicrotypeSet[protrusion]{basicmath} % disable protrusion for tt fonts
}{}
\usepackage[margin=1in]{geometry}
\usepackage{hyperref}
\hypersetup{unicode=true,
            pdftitle={Ecuaciones estructurales (SEM) para indicadores sociales},
            pdfauthor={Dr.~Héctor Nájera y Dr.~Curtis Huffman},
            pdfborder={0 0 0},
            breaklinks=true}
\urlstyle{same}  % don't use monospace font for urls
\usepackage{graphicx,grffile}
\makeatletter
\def\maxwidth{\ifdim\Gin@nat@width>\linewidth\linewidth\else\Gin@nat@width\fi}
\def\maxheight{\ifdim\Gin@nat@height>\textheight\textheight\else\Gin@nat@height\fi}
\makeatother
% Scale images if necessary, so that they will not overflow the page
% margins by default, and it is still possible to overwrite the defaults
% using explicit options in \includegraphics[width, height, ...]{}
\setkeys{Gin}{width=\maxwidth,height=\maxheight,keepaspectratio}
\IfFileExists{parskip.sty}{%
\usepackage{parskip}
}{% else
\setlength{\parindent}{0pt}
\setlength{\parskip}{6pt plus 2pt minus 1pt}
}
\setlength{\emergencystretch}{3em}  % prevent overfull lines
\providecommand{\tightlist}{%
  \setlength{\itemsep}{0pt}\setlength{\parskip}{0pt}}
\setcounter{secnumdepth}{0}
% Redefines (sub)paragraphs to behave more like sections
\ifx\paragraph\undefined\else
\let\oldparagraph\paragraph
\renewcommand{\paragraph}[1]{\oldparagraph{#1}\mbox{}}
\fi
\ifx\subparagraph\undefined\else
\let\oldsubparagraph\subparagraph
\renewcommand{\subparagraph}[1]{\oldsubparagraph{#1}\mbox{}}
\fi

%%% Use protect on footnotes to avoid problems with footnotes in titles
\let\rmarkdownfootnote\footnote%
\def\footnote{\protect\rmarkdownfootnote}

%%% Change title format to be more compact
\usepackage{titling}

% Create subtitle command for use in maketitle
\providecommand{\subtitle}[1]{
  \posttitle{
    \begin{center}\large#1\end{center}
    }
}

\setlength{\droptitle}{-2em}

  \title{Ecuaciones estructurales (SEM) para indicadores sociales}
    \pretitle{\vspace{\droptitle}\centering\huge}
  \posttitle{\par}
    \author{Dr.~Héctor Nájera y Dr.~Curtis Huffman}
    \preauthor{\centering\large\emph}
  \postauthor{\par}
      \predate{\centering\large\emph}
  \postdate{\par}
    \date{2019}

\linespread{1.05}

\begin{document}
\maketitle

{
\setcounter{tocdepth}{2}
\tableofcontents
}
\newpage

\hypertarget{presentacion-del-curso-ecuaciones-estructurales-sem-y-medicion-social}{%
\section{Presentación del curso: Ecuaciones Estructurales (SEM) y
medición
social}\label{presentacion-del-curso-ecuaciones-estructurales-sem-y-medicion-social}}

El objetivo del curso es desarrollar las capacidades críticas y
analíticas de los estudiantes para la producción y escrutinio de índices
sociales como pobreza, marginación, clase social, precariedad laboral,
seguridad alimentaria, derechos sociales, etc. Al final del curso la
expectativa es que los alumnos sean capaces de:

\begin{itemize}
\tightlist
\item
  Encender por qué es importante trabajar con medidas falsables en
  ciencias sociales
\item
  Identificar la diferencia entre un método de agregación y una
  metodología de escrutinio empírico
\item
  Apreciar la relevancia de la teoría de la medida para examinar índices
  sociales
\item
  Comprender los vínculos entre la teoría de la medida, variables
  latentes y ecuaciones estructurales
\item
  Entender por qué los principios de confiabilidad y validez son una
  necesidad necesaria para una calidad mínima de medición
\item
  Implementar análisis de ecuaciones estructurales de confiabilidad y
  validez usando \textbf{R-software} y \textbf{Mplus}
\item
  Interpretar los resultados de los análisis de una forma crítica
\item
  Identificar los usos apropiados e inapropiados de ecuaciones
  estructurales
\end{itemize}

\hypertarget{caracteristicas-de-las-sesiones}{%
\section{Características de las
sesiones}\label{caracteristicas-de-las-sesiones}}

Las sesiones combinan discusión, teoría y aplicación con el programa R.
Antes de cada clase, los alumnos deberán leer una selección de artículos
para su discusión en grupo. Los docentes impartirán cada sesión
(prepararán un *.ppt que subirán a github después de cada clase) y se
dedicará siempre un espacio para discusión, ejercicios en grupo y/o
implementación de análisis usando el programa R.

\hypertarget{github}{%
\subsection{Github}\label{github}}

Los docentes utilizarán esta plataforma para compartir los materiales
del curso (bibliografía, presentaciones, ejercicios). La dirección
relevante es:

\url{https://github.com/hectornajera83/SEMindicadores}

\hypertarget{lugar-y-hora}{%
\subsection{Lugar y hora}\label{lugar-y-hora}}

Miércoles. Salón 205. 4 a 7 pm.

\hypertarget{evaluacion}{%
\subsection{Evaluación}\label{evaluacion}}

Se utilizarán tres ejercicios para valorar los contenidos que los
alumnos manejan con confianza y aquellos que necesitan reforzarse.

\begin{itemize}
\item
  Dos ejercicios que combinan análisis crítico y aplicaciones.
\item
  Un ejercicio final que consiste en producir una medida válida y
  confiable.
\end{itemize}

\newpage

\hypertarget{clase-1-07082019-presentacion-descripcion-y-relevancia-del-curso}{%
\section{Clase 1 (07/08/2019) ``Presentación, descripción y relevancia
del
curso''}\label{clase-1-07082019-presentacion-descripcion-y-relevancia-del-curso}}

\hypertarget{objetivo}{%
\subsection{Objetivo}\label{objetivo}}

Introducir el curso, criterios de evaluación, forma de trabajo,
horarios, uso de equipo de cómputo y software. La segunda parte de la
clase se enfoca en las siguientes pregunta: ¿Por québ es importante un
curso de medición desde el punto de vista académico y de política
pública? Se subrayará la relevancia del curso haciendo referencia a
algunas de las consecuencias de medir mal en ciencias sociales.

\hypertarget{dinamica}{%
\subsection{Dinámica}\label{dinamica}}

La clase se divide en tres partes.

\begin{enumerate}
\def\labelenumi{\arabic{enumi}.}
\item
  Los docentes explicarán por qué el curso se llama así, cuáles son sus
  los objetivos del mismo, características de las sesiones, expectativas
  y forma de evaluación. \emph{Responsable: Curtis y Héctor}
\item
  Se hará una breve discusión para conocer por qué los alumnos se
  interesaron en el curso y su familiaridad con los temas del curso
  (medición, falsación en ciencias, replicabilidad, uso de programas
  estadísticos). \emph{Responsable: Curtis y Héctor}. Las siguientes
  preguntas son de referencia:
\end{enumerate}

¿Qué entienden los alumnos por \emph{medida}? ¿Si tuvieran qué medir un
concepto como seguridad alimentaria qué harían? ¿Si alguien propone un
índice alternativo cómo sabrían cuál es mejor?

\begin{enumerate}
\def\labelenumi{\arabic{enumi}.}
\setcounter{enumi}{2}
\tightlist
\item
  Los docentes hablarán sobre por qué la medición es importante desde el
  punto de vista académico (discusiones interminables y discursivas
  sobre qué medida es mejor, análisis espurios con medidas espurias) y
  práctico (problemas para la formulación de política).
  \emph{Responsable: Curtis y Héctor}
\end{enumerate}

\begin{itemize}
\tightlist
\item
  La medición es el corazón del quehacer científico
\item
  Malas medidas llevan a malas conclusiones
\item
  Ejemplo: Poca replicabilidad en medición de pobreza
\end{itemize}

\hypertarget{lecturas-para-la-siguiente-sesion}{%
\subsection{Lecturas para la siguiente
sesión}\label{lecturas-para-la-siguiente-sesion}}

Hanson (1958)

Duhem (1991)

Loken and Gelman (2017)

\hypertarget{clase-2-14082019-observacion-medicion-y-error}{%
\section{Clase 2 (14/08/2019) ``Observación, medición y
error''}\label{clase-2-14082019-observacion-medicion-y-error}}

\hypertarget{objetivo-1}{%
\subsection{Objetivo}\label{objetivo-1}}

En esta sesión se introducen y discuten los conceptos de
\emph{observación, objetividad y medición} en ciencias. Se pondrá
énfasis en la crítica a la medición objetiva, directa y perfecta y en
las implicaciones que esto tiene para producir indicadores y medidas en
ciencias sociales.

\hypertarget{dinamica-1}{%
\subsection{Dinámica}\label{dinamica-1}}

Los docentes presentarán las ideas y conceptos básicos y motivarán la
discusión grupal sobre los puntos críticos del tema de la clase. Se
pondrán ejemplos para motivar la discusión entre los alumnos. La última
parte de la sesión conectará los obstáculos en la observación con las
crisis actuales en el quehacer científico en materia de medición.

\hypertarget{lecturas-para-esta-sesion}{%
\subsection{Lecturas para esta sesión}\label{lecturas-para-esta-sesion}}

Hanson (1958)

Duhem (1991)

Loken and Gelman (2017)

\hypertarget{lecturas-para-siguiente-sesion}{%
\subsection{Lecturas para siguiente
sesión}\label{lecturas-para-siguiente-sesion}}

Kvalheim (2012)

Michell (2015)

\hypertarget{clase-3-21082019-nociones-basicas-de-medicion.-teoria-de-la-medida-variables-latentes-y-ecuaciones-estructurales}{%
\section{Clase 3 (21/08/2019) ``Nociones básicas de medición. Teoría de
la medida: Variables latentes y ecuaciones
estructurales''}\label{clase-3-21082019-nociones-basicas-de-medicion.-teoria-de-la-medida-variables-latentes-y-ecuaciones-estructurales}}

\hypertarget{objetivo-2}{%
\subsection{Objetivo}\label{objetivo-2}}

Esta clase introduce las obstáculos y preguntas que dieron origen a la
teoría de la medida y describe su evolución. La teoría de la medida
tiene más de un siglo y sin embargo hay áreas que se resisten a
implementarla. Se impartirán algunas nociones básicas de medición en
ciencias sociales -ordenamiento y clasificación de grupos- y se mostrará
cómo la historia de la teoría de la medida sea ha incorporado en los
marcos modernos de variables latentes y de ecuaciones estructurales.

\hypertarget{dinamica-2}{%
\subsection{Dinámica}\label{dinamica-2}}

Los docentes relatarán los episodios claves de la historia del
desarrollo de la teoría de la medida y pondrán énfasis en algunos
conceptos claves: error, proxy, variable latente, variables,
ordenamiento poblacional, indicadores e índices. La discusión girará en
torno a estos conceptos a fin de que los alumnos se familiaricen con
estos términos.

\hypertarget{lecturas-para-esta-sesion-1}{%
\subsection{Lecturas para esta
sesión}\label{lecturas-para-esta-sesion-1}}

Kvalheim (2012)

Michell (2015)

Bandalos (2018) (Capítulo 1) o Thorndike and Hagen (1969) (Capítulo 1)

DeVellis (2017) (Capítulo 2) **Solicitar

McDonald (1999) (Capitulo 4)

\hypertarget{lecturas-para-siguiente-sesion-1}{%
\subsection{Lecturas para siguiente
sesión}\label{lecturas-para-siguiente-sesion-1}}

Bandalos (2018) (Capítulo 1)

Brennan (2006) (Primera sección: Theory and General Principles)
**Solicitar

Cudeck and MacCallum (2012) (Introducción)

Furr (2018) (Capítulo 2) **Solicitar

McDonald (1999) (Capitulo 4)

\hypertarget{clase-4-28082019-principios-practicas-y-estandares-cientificos-para-la-medicion}{%
\section{Clase 4 (28/08/2019) ``Principios, prácticas y estándares
científicos para la
medición''}\label{clase-4-28082019-principios-practicas-y-estandares-cientificos-para-la-medicion}}

\hypertarget{objetivo-3}{%
\subsection{Objetivo}\label{objetivo-3}}

Introducir y discutir algunos conceptos básicos en medición como: pasos
para producción de una escala/índice, métricas de indicadores, unidades
de medida y agregación. Nos enfocaremos en preguntas claves como: ¿Por
qué un índice y no indicadores sueltos? ¿Qué es un índice? Después
hablaremos de cómo saber si el índice resultante es bueno, malo o
regular a fin de establecer un puente con los estándares de validez y
confiabilidad.

\hypertarget{dinamica-3}{%
\subsection{Dinámica}\label{dinamica-3}}

Los docentes brindarán una introducción y definición de los principales
conceptos y se harán algunos ejercicios en grupo para que los
estudiantes se familiaricen con estos términos.

\hypertarget{lecturas-para-esta-sesion-2}{%
\subsection{Lecturas para esta
sesión}\label{lecturas-para-esta-sesion-2}}

Bandalos (2018) (Capítulo 1)

Brennan (2006) (Primera sección: Theory and General Principles)
**Solicitar

Cudeck and MacCallum (2012) (Introducción)

Furr (2018) (Capítulo 2) **Solicitar

\hypertarget{lecturas-para-siguiente-sesion-2}{%
\subsection{Lecturas para siguiente
sesión}\label{lecturas-para-siguiente-sesion-2}}

Bandalos (2018) (Capítulo 3)

Thorndike and Hagen (1969) (Capítulo 4)

Novick (1966) (Axiomas de la teoría clásica)

Revelle (2009) (libro abierto en línea)

Revelle and Zinbarg (2009)

McDonald (1999) (Capitulo 4)

\hypertarget{clase-5-04092019-confiabilidad-teoria-del-test-clasico-y-estimadores-de-confiabilidad}{%
\section{Clase 5 (04/09/2019) ``Confiabilidad: Teoría del test clásico y
estimadores de
confiabilidad''}\label{clase-5-04092019-confiabilidad-teoria-del-test-clasico-y-estimadores-de-confiabilidad}}

\hypertarget{objetivo-4}{%
\subsection{Objetivo}\label{objetivo-4}}

El principio de confiabilidad es central en medición científica. Sin
confiabilidad, no hay validez. La teoría clásica del test propone que
toda medida observada es una versión imperfecta del parámetro real.
Existen varias propuestas para calcular la confiabilidad de una escala,
hablaremos de sus ventajas y desventajas.

\hypertarget{dinamica-4}{%
\subsection{Dinámica}\label{dinamica-4}}

Los docentes establecerán algunas nociones básicas para entender el
principio de confiabilidad, describirán formalmente la teoría y los
estimadores de confiabilidad. Se espera discutir con los alumnos algunos
elementos fundamentales como consistencia, homogeneidad y error
aleatorio.

\hypertarget{lecturas-para-esta-sesion-3}{%
\subsection{Lecturas para esta
sesión}\label{lecturas-para-esta-sesion-3}}

Bandalos (2018) (Capítulo 3)

Thorndike and Hagen (1969) (Capítulo 4)

Novick (1966) (Axiomas de la teoría clásica)

Revelle (2009) (libro abierto en línea)

Revelle and Zinbarg (2009)

\hypertarget{lecturas-para-siguiente-sesion-3}{%
\subsection{Lecturas para siguiente
sesión}\label{lecturas-para-siguiente-sesion-3}}

Revelle and Zinbarg (2009)

Zinbarg et al. (2005)

\hypertarget{clase-6-11092019-confiabilidad-teoria-de-variables-latentes-y-ecuaciones-estructurales}{%
\section{Clase 6 (11/09/2019) ``Confiabilidad: Teoría de variables
latentes y ecuaciones
estructurales''}\label{clase-6-11092019-confiabilidad-teoria-de-variables-latentes-y-ecuaciones-estructurales}}

\hypertarget{clase-7-18092019-confiabilidad-practica.-estadisticos-clasicos-y-modernos-de-confiabilidad}{%
\section{Clase 7 (18/09/2019) ``Confiabilidad: Práctica. Estadísticos
clásicos y modernos de
confiabilidad''}\label{clase-7-18092019-confiabilidad-practica.-estadisticos-clasicos-y-modernos-de-confiabilidad}}

\hypertarget{clase-8-25092019-confiabilidad-practica-con-datos-reales}{%
\section{Clase 8 (25/09/2019) ``Confiabilidad: Práctica con datos
reales''}\label{clase-8-25092019-confiabilidad-practica-con-datos-reales}}

\hypertarget{clase-9-02102019-ejercicio-1-analisis-de-confiabilidad}{%
\section{Clase 9 (02/10/2019) ``Ejercicio 1: Análisis de
confiabilidad''}\label{clase-9-02102019-ejercicio-1-analisis-de-confiabilidad}}

\hypertarget{clase-10-09102019-validez-teoria---tipos-de-validez}{%
\section{Clase 10 (09/10/2019) ``Validez: Teoría - Tipos de
validez''}\label{clase-10-09102019-validez-teoria---tipos-de-validez}}

\hypertarget{clase-11-16102019-validez-de-constructo-y-criterio-en-el-marco-de-ecuaciones-estructurales}{%
\section{Clase 11 (16/10/2019) ``Validez de constructo y criterio en el
marco de ecuaciones
estructurales''}\label{clase-11-16102019-validez-de-constructo-y-criterio-en-el-marco-de-ecuaciones-estructurales}}

\hypertarget{clase-12-23102019-validez-practica.-analisis-de-constructo-y-criterio}{%
\section{Clase 12 (23/10/2019) ``Validez: Práctica. Análisis de
constructo y
criterio''}\label{clase-12-23102019-validez-practica.-analisis-de-constructo-y-criterio}}

\hypertarget{clase-13-30102019-validez-practica-con-datos-reales}{%
\section{Clase 13 (30/10/2019) ``Validez: Práctica con datos
reales''}\label{clase-13-30102019-validez-practica-con-datos-reales}}

\hypertarget{clase-14-06112019-ejercicio-2-analisis-de-validez}{%
\section{Clase 14 (06/11/2019) ``Ejercicio 2: Análisis de
validez''}\label{clase-14-06112019-ejercicio-2-analisis-de-validez}}

\hypertarget{clase-15-13112019-medicion-invariante-comparabilidad-de-medidas-en-el-tiempo-y-grupos}{%
\section{Clase 15 (13/11/2019) ``Medición invariante: Comparabilidad de
medidas en el tiempo y
grupos''}\label{clase-15-13112019-medicion-invariante-comparabilidad-de-medidas-en-el-tiempo-y-grupos}}

\hypertarget{clase-16-20112019-medicion-invariante-practica}{%
\section{Clase 16 (20/11/2019) ``Medición invariante:
Práctica''}\label{clase-16-20112019-medicion-invariante-practica}}

\hypertarget{clase-17-27112019-planteamiento-del-ejercicio-final}{%
\section{Clase 17 (27/11/2019) ``Planteamiento del ejercicio
final''}\label{clase-17-27112019-planteamiento-del-ejercicio-final}}

\hypertarget{clase-18-04122019-revision-del-ejercicio-final}{%
\section{Clase 18 (04/12/2019) ``Revisión del ejercicio
final''}\label{clase-18-04122019-revision-del-ejercicio-final}}

\hypertarget{references}{%
\section{References}\label{references}}

\newpage

\hypertarget{refs}{}
\leavevmode\hypertarget{ref-Bandalos2018}{}%
Bandalos, Deborah L. 2018. \emph{Measurement Theory and Applications for
the Social Sciences}. Guilford Publications.

\leavevmode\hypertarget{ref-Brennan2006}{}%
Brennan, Robert L. 2006. \emph{Educational Measurement. ACE/Praeger
Series on Higher Education.} ERIC.

\leavevmode\hypertarget{ref-Cudeck2012}{}%
Cudeck, Robert, and Robert C. MacCallum. 2012. \emph{Factor Analysis at
100: Historical Developments and Future Directions}. Routledge.

\leavevmode\hypertarget{ref-DeVellis2017}{}%
DeVellis, Robert. 2017. \emph{Scale Development: Theory and
Applications}. SAGE Publications.

\leavevmode\hypertarget{ref-Duhem1991}{}%
Duhem, Pierre Maurice Marie. 1991. \emph{The Aim and Structure of
Physical Theory}. Vol. 13. Princeton University Press.

\leavevmode\hypertarget{ref-Furr2018}{}%
Furr, Michael. 2018. \emph{Psychometrics: An Introduction}. SAGE
Publications.

\leavevmode\hypertarget{ref-Hanson1958}{}%
Hanson, Norwood Russell. 1958. \emph{Patterns of Discovery: An Inquiry
into the Conceptual Foundations of Science}. Vol. 251. CUP Archive.

\leavevmode\hypertarget{ref-Kvalheim2012}{}%
Kvalheim, Olav M. 2012. ``History, Philosophy and Mathematical Basis of
the Latent Variable Approach: From a Peculiarity in Psychology to a
General Method for Analysis of Multivariate Data.'' \emph{Journal of
Chemometrics} 26 (6): 210--17. \url{https://doi.org/10.1002/cem.2427}.

\leavevmode\hypertarget{ref-Loken2017}{}%
Loken, Eric, and Andrew Gelman. 2017. ``Measurement Error and the
Replication Crisis.'' \emph{Science} 355 (6325): 584--85.
\url{https://doi.org/10.1126/science.aal3618}.

\leavevmode\hypertarget{ref-McDonald1999}{}%
McDonald, R. P. 1999. \emph{Test Theory: A Unified Treatment}. Edited by
R. P. McDonald. Mahwah, N.J. L. Erlbaum Associates.

\leavevmode\hypertarget{ref-Michell2015}{}%
Michell, Joel. 2015. ``Measurement Theory: History and Philosophy.'' In
\emph{International Encyclopedia of the Social \& Behavioral Sciences
(Second Edition)}, edited by James D. Wright, Second Edition, 868--72.
Oxford: Elsevier.
\url{https://doi.org/https://doi.org/10.1016/B978-0-08-097086-8.43062-X}.

\leavevmode\hypertarget{ref-Novick1966}{}%
Novick, Melvin R. 1966. ``The Axioms and Principal Results of Classical
Test Theory.'' \emph{Journal of Mathematical Psychology} 3 (1): 1--18.
\url{https://doi.org/https://doi.org/10.1016/0022-2496(66)90002-2}.

\leavevmode\hypertarget{ref-Revelle2009a}{}%
Revelle, William. 2009. ``An Introduction to Psychometric Theory with
Applications in R.'' Springer.

\leavevmode\hypertarget{ref-Revelle2009}{}%
Revelle, William, and Richard. Zinbarg. 2009. ``Coefficients Alpha,
Beta, Omega, and the Glb: Comments on Sijtsma.'' \emph{Psychometrika} 74
(1): 145--54. \url{https://doi.org/10.1007/s11336-008-9102-z}.

\leavevmode\hypertarget{ref-Thorndike1969}{}%
Thorndike, R., and Elizabeth Hagen. 1969. ``Measurement and Evaluation
in Education and Psychology.'' New York, NY: John Wiley; Sons.

\leavevmode\hypertarget{ref-Zinbarg2005}{}%
Zinbarg, RichardE., William Revelle, Iftah Yovel, and Wen Li. 2005.
``Cronbach's \(\alpha\), Revelle's \(\beta\), and Mcdonald's
\(\omega_h\) : Their Relations with Each Other and Two Alternative
Conceptualizations of Reliability.'' \emph{Psychometrika} 70 (1):
123--33. \url{https://doi.org/10.1007/s11336-003-0974-7}.


\end{document}
